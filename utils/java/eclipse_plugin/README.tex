\documentclass[10pt]{article}
\usepackage[top=1cm, bottom=1cm, left=1cm, right=1cm]{geometry} 
\usepackage[utf8x]{inputenc}

%opening
\title{Eclipse UMC Plugin Readme}
\author{Timotei Dolean}

\begin{document}

\maketitle

\newcounter{cnt}
\newcommand{\icnt}{ \stepcounter{cnt} \thecnt }

\section{Common prerequisites}
\icnt) Download and install Eclipse ({\bf http://eclipse.org/downloads/}) \\
\icnt) Install the library "EMF". \\ 
For that, in Eclipse: Help menu -  Install new software, insert the link \\
({\bf http://download.eclipse.org/modeling/emf/updates/releases/}) in that textbox, and press enter. 
Check the "Group items by category", and select: \\ 
    - {\bf "EMF SDK 2.5.0 (EMF + XDS)} - select "EMF - Eclipse Modeling Framework Core Runtime"\\
\icnt) Install the library "XText". \\
The download link: http://download.itemis.com/updates/releases. 
After that, select the latest version of the following packages and 
install them (Don't forget to have checked the "Group items by category" checkbox): \\
    - {\bf "TMF XTEXT SDK (incubation) x.x.x"} \\
    - {\bf "Xtext antlr Support x.x.x"} \\

\section{Developer}
\subsection{Setup the environment}
\setcounter{cnt}{0}
\icnt) Checkout plugin's folder from the svn (http://svn.gna.org/svn/wesnoth/trunk/utils/java/) \\
\icnt) In Eclipse, right click in Package Explorer/Project Navigator and then select 
Import - General- Existing projects into Workspace \\
\icnt) Select the path where you downloaded the java folder, and check all the 4 projects 
(the eclipse plugin + the other 3 xtext projects) \\
\icnt) Build the projects. 
\subsection{Running the plugin}
After you've setup the environment and built the plugin you can run it. \\
\setcounter{cnt}{0}
\icnt) Open the file plugin.xml \\
\icnt) In the {\bf Testing} section, select the desired method of launching the plugin 
(non-debug/debug mode).

\section{User}
\subsection{Installing the plugin}
\setcounter{cnt}{0}
\icnt) Install the plugin from ({\bf http://eclipse.wesnoth.org/ } \\
\icnt) Select {\bf "Wesnoth UMC Plugin"} and press finish.

\section{Everybody}
\subsection{Using the plugin}
Ok. So, after you have your plugin installed(user) or running(developer), you can use its features. 
But before of all, you must update the preferences. For this, go in "Window - Preferences - Wesnoth UMC Plugin".
Here you should set all options. Should the working directory be empty, it will be computed automatically 
based on wesnoth's executable

\subsubsection{Wizards}
To create a new {\bf Campaign}, open the "New..." menu (either from File - New menu, or right click in the 
project navigator and select "New..." ). After that select "Wesnoth Campaign". Fill in the information needed,
and press finish. Your campaign project is created in the workspace. \\
To create a new {\bf Scenario}, open the "New..." menu, and select "Wesnoth scenario". 
Complete the information needed and press finish.

\subsubsection{Menus}
There are currently 2 types of menus: the context menus for different file/folder types and the toolbar menus. \\

{\bf Project context menus} - right click on the campaign projects created with the plugin\\
   {\it Wesnoth project report} - will show a simple report with the numer of maps, scenarios and units.

{\bf .cfg files context menus} - right click on any .cfg file\\
   {\it Open scenario in game} - opens the selected file's scenario (if it contains one) in wesnoth \\
   {\it WML Tools} - provides some options for using the wmltools with the specified file 
   (e.g. run wmllint against the file and see the output in the console) \\
   {\it Preprocessor} - provides ways of preprocessing and showing the result in an editor inside eclipse.

{\bf "maps" folder} - right click on the "maps" folder\\
   {\it Import map} - Shows a file selection window that let's you select a .map file that will be copied in your campaign project.


\end{document}
