\documentclass[a4paper,12pt,english]{article}

\newcommand{\ana}{\textbf{ana}}

\usepackage{listings}

% \usepackage{setspace}

\usepackage{color}
\usepackage{textcomp}
\definecolor{listinggray}{gray}{0.9}
\definecolor{lbcolor}{rgb}{0.95,0.95,0.95}
\lstset{
    backgroundcolor=\color{lbcolor},
    tabsize=4,
    rulecolor=,
    language=matlab,
        basicstyle=\scriptsize,
        upquote=true,
        aboveskip={1.5\baselineskip},
        columns=fixed,
        showstringspaces=false,
        extendedchars=true,
        breaklines=true,
        prebreak = \raisebox{0ex}[0ex][0ex]{\ensuremath{\hookleftarrow}},
        frame=single,
        showtabs=false,
        showspaces=false,
        showstringspaces=false,
        identifierstyle=\ttfamily,
        keywordstyle=\color[rgb]{0.1,0.1,0.6}\bfseries,
        commentstyle=\color[rgb]{0.133,0.545,0.133},
        stringstyle=\color[rgb]{0.627,0.126,0.941},
}


\title{Developing Server/Client applications \\
using the \ana \ API/library}

\author{Guillermo Biset}

\begin{document}

\maketitle

\vfill

\begin{abstract}
Developing network applications is no easy task. Most of the time, the
important features are not  even network-related, yet the implementers
spend most of their time coding  to get this part of the functionality
right.

This  document introduces  \ana, an  API designed  to let  you develop
simple network  server and  client applications using  an asynchronous
(i.e. non blocking) model in  C++ without having to deal directly with
network components such as sockets.

\end{abstract}

\newpage

\tableofcontents

\newpage

\section{Preliminaries}

\textbf{Disclaimer}: This section contains no technical information
whatsoever.

\section{Introduction}

\textbf{Ana} (stands for Asynchronous  Network API) is an API designed
to help you develop network applications with server and client parts.
To implement  this functionality, the  API uses an  asynchronous model
(i.e.  non-blocking operations .) This way, the progammer doesn't need
to  pay  too  much  attention  to  the network  code  itself  but  can
concentrate on other features relevant to the program.

Also, \ana \ comes with  an accompanying implementation.  Hence, it is
also a library that implements  the API features.  For example, MPI is
an API  for message passing and  LAM/MPI is a  library that implements
it. Further on it is possible that the two will be separated, but this
is unlikely at the moment.

The dynamics involved in writing server/client applications using \ana
\ is  fairly simple.   There are interfaces  designed for each  set of
handlers relevant to network events (described in later sections) that
one must implement  and every event will be  issued by a corresponding
component with a unique ID.

In the simpler cases, developing a server involves:
\begin{enumerate}
    \item Creating an \ana::server object.
    \item Running it on a given port.
    \item Having  a class of  your own inherit from  the corresponding
      interfaces to handle network events.
    \item Implement these handlers.
\end{enumerate}

The client side implementation is analogous:
\begin{enumerate}
    \item Create an \ana::client object.
    \item Instruct it to connect to a given address:port.
    \item   Create  classes  that   inherit  from   the  corresponding
      interfaces to handle network events.
    \item Implement these handlers.
\end{enumerate}

\section{On Asynchronous Operations}

Evidently, implementing these handlers  may not be so simple. However,
a  large   amount  of  functionality  may  be   achieved  from  simple
implementations in them.

Every  operation in  \ana  \ is  asynchronous,  which is  both a  good
feature and a  bad one. The drawback is  that some assumptions usually
made about function invocation require thinking twice about them.

The main problem is not understanding fully what the function does, in
our  case, and  as  an  example, a  \texttt{send}  operation does  not
necessarily  return after  it has  sent any  data to  anyone,  it just
\emph{queues} the corresponding operation, thus leaving the work to be
carried out at a later time.

It is  through the invocation of  the handler that  one can understand
what happened.

For example:
\begin{itemize}
    \item Returning  from a send  operation doesn't mean the  data was
      sent (you just started sending it, the library will let you know
      when the operation was completed.)
    \item Returning  from a connect operation doesn't  mean the client
      succesfully connected  to the server (you just  instructed it to
      start a connection attempt.)
    \item  Implementing  the  handler  for  an  event  involves  error
      checking  to see  if  the  event failed  and,  if you  performed
      several operations of a  single type (e.g. many send operations)
      you won't be  able to identify which of  these failed (since the
      error belongs only to \textit{one} of these.)
\end{itemize}

%\section{Features}

%\textbf{Ana} incorporates the following features:
%\begin{description}
% \item [
%\end{description}

\section{API overview}

\subsection{Namespaces}

The list of namespaces is as follows:

\begin{description}

\item[\ana] : Main namespace.

\item[\ana::time]   :  Functions  to   build   time  durations
  (i.e. \ana::time::seconds.)

\item[\ana::detail] : Implementation specific definitions. 

\end{description}

The whole  API can be found under  one namespace: \ana. From  now on I
will  refer  to elements  in  the \ana  \  namespace  using C++  scope
operator ::, for instance, method \texttt{f} from namespace \ana \ may
be referred to as \ana::\texttt{f}.

Since  timeout  configuration (to  be  discussed)  may  be done  using
different  functions  to  construct  time durations,  a  \textbf{time}
namespace  is added  to  \ana. So,  the \ana::\textbf{time}  namespace
contains functions for creating time durations.

The last namespace, and the one  least relevant to the user of the API
is the namespace \textbf{detail}, enclosing all implementation related
code.

\subsection{Basic Types Defined}

Here is  a list  of some of  the simple  types defined under  the \ana
\ namespace

\begin{description}
   \item [ana\_uint] : Standard unsigned int. This type should be used
     if you  expect to serialize data  in an object, send  it over the
     network, and then extract it on  the other side (which may have a
     different architecture.)
   \item [ana\_int] :  Standard int. Idem.
   \item [client\_id] : A unique  ID of a network component. This type
     has a  total order, so it  can be compared to  other instances of
     its type. A server will have an ID equal to zero and every client
     a greater than zero value.
   \item [port] : A string representing a port  descriptor, the string
     should be able to cast itself to a unsigned short int.
   \item [address] : A string  representing a network  address, it may 
     be an IP address of a hostname.
   \item [send\_type]  : Send operation  type, this is related  to the
     lifecycle of  the buffer  used in the  send operation,  which may
     involve  copying the  buffer or  not  (this can  result in  great
     improvement regarding memory usage.)  A value of this type may be
     evaluated  as a  boolean  (where \emph{true}  means  to copy  the
     buffer.)   To  set   a  value,  constants  \texttt{ZeroCopy}  and
     \texttt{CopyBuffer} are provided.   See section \ref{bufcopy} for
     more information.
   \item  [error\_code] :  Descriptor of  an error  code. It  can also
     evaluate  to a boolean  value where  \emph{false} means  no error
     occurred.
\end{description}

\subsection{Main Classes}

The main classes to be used from the \ana \ API are:
\begin{description}

\item[ana::server] :  A network  server.  An object  of this  type can
  handle several connected clients.  It is possible to create pointers
  to    an    object    of    this    type    using    the    function
  \ana::\texttt{create\_server}.

\item[ana::client]  : A  network  client. A  network  entity that  can
  connect to  a running server.  To create a  pointer to an  object of
  this type use the \ana::\texttt{create\_client} function.

\item[ana::timer] :  An asynchronous deadline  timer that can  be used
  for general  purposes.
\end{description}

\subsection{Main Interfaces}

\begin{description}

\item[\ana::listener\_handler]  :  Should  be  implemented  to  handle
  incoming messages and the disconnection of a connected component.

\item[\ana::connection\_handler]  : Handling  of  new connections.

\item[\ana::send\_handler] : Handling completed or failed send events.

\end{description}

\subsection{Extras}

\subsubsection{Timers}

Besides  supporting  configurable timeouts  on  send operations,  \ana
\ provides a  simple timer interface to be  used for general purposes.
One can create timers and set  them to call a specific handler after a
certain amount of time has passed.

To  create  these  time   lapses  functions  are  provided  under  the
\textbf{time} namespace for  milliseconds, seconds, minutes and hours,
and these functions  may be used in both  cases (instructing a general
purpose  timer to wait  for given  amount of  time or  configuring the
amount of time required for a timeout event on a send operation.)

Here are a few examples of how to construct time durations:
\begin{center}
\begin{tabular}{|l|l|}
  \hline
  Five minutes & \texttt{ana::time::minutes(5)} \\
  \hline
  A second and a half & \texttt{ana::time::seconds(1.5)} \\
  \hline
  One hour and ten milliseconds & \texttt{ana::time::hours(1)} + \\
  \ &     \texttt{ana::time::milliseconds(10)} \\
  \hline
\end{tabular}
\end{center}

There are  two ways  of creating  timers, via an  \ana \  component or
using the type directly. You should always try to use ana's server and
client     objects    to    create     them    via     the    function
\ana::\texttt{create\_timer}.

The easy way is using  the type directly to declare regular variables.
However, this  burdens the  memory greatly in  comparison of  using an
\ana \  component (either  server or client)  to create the  timer for
you.

\section{Using the API}

\subsection{Important Recommendations}

Here   are   some   very   important   recommendations   about   using
\ana. Explanations can be found in the following subsection.

\begin{itemize}
   \item If you are unsure about the consistency of the memory region
     you are using to send data, then use a send operation that copies
     the buffer before using it (this ensures that the buffer is
     copied before the invocation to the operation returns.)
     
     There are two ways to  accomplish this, but to keep things short,
     just use the default send operation without setting that specific
     parameter (e.g. \texttt{server\_->send\_all(  ana::buffer ( str )
       ) } .)
   \item   Always  prefer   single   multiple-send  operations   (e.g.
     \texttt{send\_if})   to  executing   many  times   a  single-send
     operation (\texttt{send\_one}.)
   \item  If you  are \textbf{absolutely}  sure the  memory  you'll be
     using for the send buffer will be consistent from the time of the
     invocation  to  the  send  operation  to the  invocation  of  its
     associated  handler,   then  use  a  zero   copy  send  operation
     (e.g.    \texttt{server\_->send\_all(    ana::buffer(   str    ),
       ana::ZeroCopy ) } .)
   \item  Use  the  code as  if  handlers  are  called in  a  mutually
     exclusive way,  but they are executed concurrently  with the rest
     of your application code.
\end{itemize}

\subsection{Details about Buffers}
\label{bufcopy}

There is a reason why  multiple-send operations are better. If you are
using     a    multiple-send    operation     (    \texttt{send\_all},
\texttt{send\_if} ) that copies the buffer (the default behavior) then
the buffer will be copied only once. Also, sending to each client will
read from that original buffer. Finally, the buffer will be destructed
automatically after the last handler has been called.

\subsection{Details about Threads}

\textbf{Ana} can't intercede between you and the rest of your program,
this means that it can't  ensure mutual exclusion between its code and
your own. What it can ensure is that the handlers you implemented will
be  called from a  single thread  (i.e. there  won't be  two different
handlers running concurrently.)

But  this  forces  the  user  of  the API  to  take  this  case  under
consideration    and    implement    its    own    mutual    exclusion
mechanism.  However,  the  most  common  scenario will  have  a  clear
separation between the rest of the program and the code using \ana.

For instance, there is no shared data structure between \ana \ and its
user.  The case  to look  out for  is when  the implementation  of the
network event  handlers happen  to modify data  that is  also modified
within the rest of the code.

\section{Example: Chat Application}

This  section describes  the  devolopment  of a chat server and client
applications. There is  nothing  interesting  about it except the fact
that it serves to show how one would go about using \ana.

First of all, we will  develop  two  separate executable files, one for
the server and one for the client. The server application  will start a 
chat server in a given  port  and  connect clients, keeping track of their
assigned  names. It also provides commands to change a client's name and
listing everyone connected.

A client can connect to a running server and send  messages or commands
to it. Regular messages will in  turn  be  broadcasted  to  every other
client (prepending the name of the originating client first) and commands
executed.

Since the client application is simpler, lets look at code for it first.

\subsection{Client Code}

The main function is very straightforward, it just creates a \texttt{ChatClient}
object and tells it to run according to parameters taken from the command
line. So, if the executable is \texttt{client} one may execute the
following command: \texttt{./client -n billy -a address.com -p 12345}.

\begin{table}[!htb]
\lstset{language=C++}
\begin{lstlisting}[frame=single]
int main(int argc, char **argv)
{
    GetOpt_pp options(argc, argv);

    if (options >> OptionPresent('h', "help"))
        show_help();
    else
    {
        port pt(DEFAULT_PORT);
        std::string address(DEFAULT_ADDRESS);
        std::string name("Unnamed");

        options >> Option('p', "port", pt) 
                >> Option('a',"address",address) 
                >> Option('n',"name",name);
        std::cout << "Main client app.: Starting client" << std::endl;

        ChatClient client(address,pt,name);
        client.run();
    }
}
\end{lstlisting}
\centering \caption{The client's main function.} 
\label{client-main}
\end{table}

So, not much to look at here, move along. Table \ref{chatclient-decl} is a little
more interesting and shows how one should declare classes to implement the
handler methods relevant to network events. 

\begin{table}[!htb]
\lstset{language=C++}
\begin{lstlisting}[frame=single]
class ChatClient : public ana::listener_handler,
                   public ana::send_handler,
                   public ana::connection_handler
{
    public:
        ChatClient(const std::string& address, port pt, std::string name)
          : continue_(true),
            client_( create_client(address,pt) ),
            name_(name)
        {
        }
        
        void run();
        /* ... */
    private:
        /* Handler profiles. */
        virtual void handle_connect( ana::error_code error, 
                                     ana::client_id server_id );

        virtual void handle_disconnect( ana::error_code error, 
                                        ana::client_id server_id);

        virtual void handle_message( ana::error_code error, 
                                     ana::client_id, 
                                     ana::detail::read_buffer msg);

        virtual void handle_send( ana::error_code error, 
                                  ana::client_id client);

        /* Attributes.       */
        bool                continue_;
        ana::client*        client_;
        std::string         name_;
};
\end{lstlisting}
\centering \caption{A class that will implement handlers (all of them in this case.)} 
\label{chatclient-decl}
\end{table}

\begin{table}[!htb]
\lstset{language=C++}
\begin{lstlisting}[frame=single]
void ChatClient::run()
{
    try
    {
        client_->connect( this );
        client_->set_listener_handler( this );
        client_->run();

        // list available commands

        run_input(); // read lines and send them
    }
    catch(const std::exception& e)
    {
        std::cerr << e.what() << std::endl;
    }

    delete client_;
}
\end{lstlisting}
\centering \caption{Running a client application.} 
\label{client-run}
\end{table}

For this particular client application most handlers are trivial, method
\texttt{handle\_disconnect} simply informs that the server disconnected
and sets things up to quit, function \texttt{handle\_message} prints the
incoming message and \texttt{handle\_send} reports the error if it occurred.

Method \texttt{handle\_connect} has some interest to it since it tells the
server what its desired name is. Table \ref{client-connect} shows the
corresponding code.

\begin{table}[!htb]
\lstset{language=C++}
\begin{lstlisting}[frame=single]
virtual void handle_connect( ana::error_code error, client_id server_id )
{
  if ( error )
      std::cerr << "Error connecting." << std::endl;
  else
      client_->send( ana::buffer( std::string("/name ") + name_) , this);
}
\end{lstlisting}
\centering \caption{Implementation of the connection handler in the chat client's app.} 
\label{client-connect}
\end{table}

\subsection{Server Code}

Things are a little more interesting on the server side. Table \ref{server-run}
shows the code from running a server and table \ref{server-message} shows the
code that handles an incoming message in the server side. Exactly as one would
expect, if not a command, the message is broadcasted to everyone else (i.e.
with the exception of the originating client.)


\begin{table}[!htb]
\lstset{language=C++}
\begin{lstlisting}[frame=single]
void ChatServer::run(port pt)
{
    server_->set_connection_handler( this );
    server_->set_listener_handler( this );
    server_->run(pt);
    server_->set_timeouts(ana::FixedTime, ana::time::seconds(10));

    std::cout << "Server running, Enter to quit." << std::endl;

    std::string s;
    std::getline(std::cin, s); //yes, i've seen better code :)
}
\end{lstlisting}
\centering \caption{Running a server.} 
\label{server-run}
\end{table} 


\begin{table}[!htb]
\lstset{language=C++}
\begin{lstlisting}[frame=single]
void ChatServer::handle_message( ana::error_code error, 
                                 client_id client, 
                                 ana::detail::read_buffer buffer)
{
    if (! error)
    {
        std::string msg = buffer->string();

        if (msg[0] == '/')
            parse_command(client, msg); //interpret the command
        else
        {
            std::stringstream ss;
            ss << names_[client] << " : " << msg;
            server_->send_if(ana::buffer( ss.str() ), this,
                     create_predicate( boost::bind( 
                         std::not_equal_to<client_id>(), client, _1) ) );
        }
    }
    else
        handle_disconnect(error, client);
}
\end{lstlisting}
\centering \caption{Conditional broadcasting of an incoming message.} 
\label{server-message}
\end{table} 


\end{document}
