\chapter{Design details}

Now that the big picture regarding the library is known, I dive into parts of
the code that can use more explanation. If a part is not explained here, it doesn't mean
the code is obvious or simple, lack of time to document it properly is more
likely the excuse.

\section{Boot strapping the library}

Before the library can be started the available widgets and windows need to
be registered. This registering is done before \textsc{main()} is started.
The registering is done by some small static classes that do the registering
of the item\footnote{Widget or Window.} in its constructor. Since this
involves some redundant typing there are REGISTER\_XXX macros written.

After all items are registered the library can be started, which is simply
done by \textsc{gui2::init()}. This function does all required steps to get
the library up and running.

\section{Layout algorithm}

An important part of the gui engine is to properly layout the widgets in the
available space. The documentation of that algorithm is written in
doxygen\footnote{\url{http://devdocs.wesnoth.org/layout\_algorithm.html}}.

\section{Event handling and dispatching}
\label{event_handling}

The event handling translates the ``raw'' SDL events to an event structure
specific to gui2, effectively decoupling the interface. This also allows adding
frameworks to fake events. The documentation of that code is written in doxygen
doxygen\footnote{\url{http://devdocs.wesnoth.org/event_dispatching.html}}

\section{Iterator}

The iterator class is written\footnote{The code hasn't been written yet, only
designed how it should look. Still I feel the design is rather finished and I
can update this paper if details change too much.} to alleviate certain
problems. The scrollbar containers have their own grid and a grid for its
content. The implantation makes the looping over all children tricky. This has
been solved, but the design of the solution is rather awkward. Obviously fixing
the design is the right thing to do, but that breaks the iteration.

By first writing an iteration class the interface for iteration can be kept
cleaner and the classes can easily be refactored.

\subsection{Design}

There are two basic kind of iterators in the design, the simple ones that can
only travel to themselves and their direct children. This type will be referred to
as basic iterator hereafter. 

This basic iterator is a superclass for several specific subclasses. The
superclass is a concrete class, which can be instantiated. This class acts as a
sentinel iterator, signalling the end of a list.

Every widget has its own creation function returning a pointer to a subclass
object, allowing the main iterator to keep pointers to basic iterators, which it
uses for travelling.

\paragraph{}

The other kind is called the main iterator. This iterator is of the type which the user
normally creates and uses. The class is a template class, where a policy designs
how the travelling should go\footnote{For now only one policy is planned, but I
can think of more kinds}.

When the main iterator is created, it's possible to add a predicate to the
constructor. The predicate determines what the travelling routine does with a
candidate widget.

Now that we know the players in the game, let's look further into the implementation
details and the decisions made.

\subsubsection{Travelling}

A widget can have several ``layers'', namely:

\begin{description}
\item[self] The widget itself.
\item[grid] Container widget has a grid which is another layer, note that for a
	grid widget, its grid and self layer are the same.
\item[content] Scrollbar containers have a grid, containing their scrollbars and
	a dummy content spacer. Their real content is stored in a separate internal
	node, which is used as the content layer.
\end{description}

The travel policy decides in which order these layers are visited, and when a
node from a ``grid'' or ``content'' is returned, which direction to travel?
Travel over them first and then into their children or children first? These
decisions are coded in the travelling policy.

While travelling, the policy finds a candidate widget to travel to. This widget is
offered for evaluation to the predicate, which returns one of the following
values:

\begin{description}
\item[return] The widget referred by the iterator is accepted and the algorithm
	found its target.
\item[continue] The widget is skipped and the next candidate is sought.
\item[break] The widget is not allowed and the travelling at this layer stops.
	This doesn't mean the algorithm gives up. This layer is cancelled but the
	travelling path might have more options, which are used. At the moment there's
	no way to tell that the candidate has failed and that the searching should stop
	altogether, an ``exit'' result might be added for that case.
\end{description}

Obviously, these names are inspired by the C++ keywords.

\subsubsection{Copying}

The basic iterators are copyable since their state can easily be copied.

\paragraph{}

The main iterator can't be copied, it would involve copying the state
and copying the main iterator is deemed not to be needed.

\subsubsection{Operator++(int) (postfix increment)}

The postfix increment operator hasn't been added to either widget type. For the
basic iterators, there's the problem that the type iterator returned after
iteration might not be the original type. This works properly on the prefix
increment operator since it's a virtual function, using the covariant return
type.

The postfix version needs to return an object and that would involve slicing the
returned object instead of binding it to a reference.

\paragraph{}

The main iterator has no postfix increment operator since it's not copyable.
This of course is a bit like the chicken and the egg problem: since I wanted to
prevent the postfix increment operator I made the class not copyable.

For most (standard) iterator classes the overhead of copying isn't too high
since those iterators carry little state.  This iterator carries a lot of state,
so you don't want to copy it. The cost of copying is $O(n)$ where $n$ is the
depth in the widget tree the iterator is\footnote{This might depend on the
travelling policy, but it won't get cheap}. So in the same spirit that the
standard library doesn't add operator[] for std::list I omitted the postfix
increment operator.


\section{Callbacks}

\Cref{event_handling} describes the generic event handling for the widgets, but
in some cases a widget wants to notify other widgets of a state change. Parts of
gui2 use simple C-style callbacks for that purpose, but using boost::function
makes better replacement. Therefore the code was analysed closer and another
problem should be rectified; The callback causes a binding between two objects,
but they are not notified of the deletion of the object leaving a hole for
calling a destroyed object.

In order to fix the problem two classes are defined:

\begin{description}
\item[tnotifier] The class sending the wanted notification.
\item[tnotifiee] The class to manage the lifetime of the connection.
\end{description}

The tnotifiee is a small class that holds a pointer to the receiver it's
connected to, upon destruction it uses this pointer to deregister itself by the
receiver. After that the callback function will no longer be called.

The class should be used as member of a class so it can manage the lifetime of
the connection with the tnotifier.

\paragraph{}

The tnotifier is the main class. When a callback is registered it stores the
callbacks in an internal list and updates the pointer in the tnotifiee to
itself.

Upon destruction it clears the pointer in all tnotifiees that point to use. This
way upon destruction of the tnotifiee it won't try to deregister itself with
this destroyed object.

Subclasses of the tnotifier should add an notification function, so the notifier
can call all callbacks in the list. The notifier should take care of this
calling.

\section{Tooltip placement}
\label{sec:tooltip_placement}


The placement of the GUI2 tooltips are handled by the tooltip window. This
window needs to determine where it wants the tooltip to be placed, and
determine its optimal size.

\begin{description}
\item[\cref{sec:tooltip_placement:normal}]
	Shows the how normal tooltips look in the game window.

\item[\cref{sec:tooltip_placement:large}]
	Shows the how large tooltips look in the game window.

\item[\cref{sec:tooltip_placement:huge}]
	Shows the how huge tooltips look in the game window.

\end{description}


\subsection{Normal}
\label{sec:tooltip_placement:normal}

This tooltip has a single line of text and the width of the tooltip is below or
equal to the wanted width of the tooltip. The wanted width of the tooltip is
determined by a size where the line-length of the text is pleasant to read.

Or the  tooltip has either a end of line character, causing it to be spread
over several lines, or its text is too long for the wanted width, causing it to
be wrapped. The text's height will still fit on the screen.

\Cref{fig:tooltip_placement} gives a visual representation of the placement
algorithms used. The algorithms used are:
\begin{enumerate}[I.]
\item\label{sec:tooltip_placement:algorithm:I}
	This is the preferred placement of the tooltip. The lower right corner of
	the tooltip is attached to the top left of the mouse pointer.

\item\label{sec:tooltip_placement:algorithm:II}
	This is used when the mouse \mbox{x-position} is greater than the width of
	the tooltip window, but the mouse \mbox{y-position} is smaller than the
	height of the tooltip window. The top of the tooltip is attached to the top
	of the screen at the left side of the mouse pointer.

\item\label{sec:tooltip_placement:algorithm:III}
	This is used when the mouse \mbox{x-position} is smaller than the width of
	the tooltip window, but the mouse \mbox{x-position} is greater than the
	height of the tooltip window. The left side of the tooltip is attached to
	the left side of the screen at the top of the mouse pointer.

\item\label{sec:tooltip_placement:algorithm:IV}
	This is used when the mouse \mbox{x-position} is smaller than the width of
	the tooltip window, and the mouse \mbox{y-position} is smaller than the
	height of the tooltip window. The top of the tooltip is attached to the top
	of the screen at the right side of the mouse pointer.

\end{enumerate}

\begin{figure}[tbh]
\centering\begin{tikzpicture}[
	  x=0.0008\textwidth
	, y=0.0008\textwidth
	, pointer/.style={dart, draw=black, rotate=120, anchor=tip, fill=gray!60}
	, shadow/.style={dart, draw=black!40, rotate=120, anchor=tip, fill=gray!15}
	, tooltip/.style={draw=black, minimum width=100, minimum height=50}
	]

	\draw (0,0) rectangle (1200, 800);

	% pointer size 55 x 75
	\node[pointer] at (1135, 80) {};
	\node[tooltip, anchor=south east] at (1135, 80) {Algorithm I};

	\node[shadow] at (1135, 795) {};
	\node[pointer] at (1135, 725) {};
	\node[shadow] at (1135, 615) {};
	\node[tooltip, anchor=north east] at (1135, 795) {Algorithm II};


	\node[shadow] at (5, 80) {};
	\node[pointer] at (155, 80) {};
	\node[shadow] at (370, 80) {};
	\node[tooltip, anchor=south west] at (5, 80) {Algorithm III};

	\node[shadow] at (5, 795) {};
	\node[pointer] at (5, 725) {};
	\node[shadow] at (5, 615) {};
	\node[tooltip, anchor=north west] at (70, 795) {Algorithm III};


\end{tikzpicture}
\caption{Overview of the tooltip placement.}
\label{fig:tooltip_placement}
\end{figure}


\subsection{Large}
\label{sec:tooltip_placement:large}

This tooltip needs so much space that its height no longer fits on the
screen. In this case the width will be increased. There is no guarantee the
tooltip will fit properly; the creator of the tooltip should consider
whether or not it contains too much information. Once the width is adjusted the
placement algorithm used it the same as for the normal tooltip, see
\cref{sec:tooltip_placement:normal}.


\subsection{Huge}
\label{sec:tooltip_placement:huge}

The tooltip will certainly not fit nicely on the screen and no attempt is
made to even try to do something sane. Instead it is placed in the top left
corner. Unless a very small screen resolution is used the creator of the
tooltip really needs to consider reducing the amount of text on the tooltip.
There is no guarantee the entire tooltip is visible.

