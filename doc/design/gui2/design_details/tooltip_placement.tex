\section{Tooltip placement}
\label{sec:tooltip_placement}


The placement of the GUI2 tooltips are handled by the tooltip window. This
window needs to determine where it wants the tooltip to be placed, and
determine its optimal size.

\begin{description}
\item[\cref{sec:tooltip_placement:normal}]
	Shows the how normal tooltips look in the game window.

\item[\cref{sec:tooltip_placement:large}]
	Shows the how large tooltips look in the game window.

\item[\cref{sec:tooltip_placement:huge}]
	Shows the how huge tooltips look in the game window.

\end{description}


\subsection{Normal}
\label{sec:tooltip_placement:normal}

This tooltip has a single line of text and the width of the tooltip is below or
equal to the wanted width of the tooltip. The wanted width of the tooltip is
determined by a size where the line-length of the text is pleasant to read.

Or the  tooltip has either a end of line character, causing it to be spread
over several lines, or its text is too long for the wanted width, causing it to
be wrapped. The text's height will still fit on the screen.

\Cref{fig:tooltip_placement} gives a visual representation of the placement
algorithms used. The algorithms used are:
\begin{enumerate}[I.]
\item\label{sec:tooltip_placement:algorithm:I}
	This is the preferred placement of the tooltip. The lower right corner of
	the tooltip is attached to the top left of the mouse pointer.

\item\label{sec:tooltip_placement:algorithm:II}
	This is used when the mouse \mbox{x-position} is greater than the width of
	the tooltip window, but the mouse \mbox{y-position} is smaller than the
	height of the tooltip window. The top of the tooltip is attached to the top
	of the screen at the left side of the mouse pointer.

\item\label{sec:tooltip_placement:algorithm:III}
	This is used when the mouse \mbox{x-position} is smaller than the width of
	the tooltip window, but the mouse \mbox{x-position} is greater than the
	height of the tooltip window. The left side of the tooltip is attached to
	the left side of the screen at the top of the mouse pointer.

\item\label{sec:tooltip_placement:algorithm:IV}
	This is used when the mouse \mbox{x-position} is smaller than the width of
	the tooltip window, and the mouse \mbox{y-position} is smaller than the
	height of the tooltip window. The top of the tooltip is attached to the top
	of the screen at the right side of the mouse pointer.

\end{enumerate}

\begin{figure}[tbh]
\centering\begin{tikzpicture}[
	  x=0.0008\textwidth
	, y=0.0008\textwidth
	, pointer/.style={dart, draw=black, rotate=120, anchor=tip, fill=gray!60}
	, shadow/.style={dart, draw=black!40, rotate=120, anchor=tip, fill=gray!15}
	, tooltip/.style={draw=black, minimum width=100, minimum height=50}
	]

	\draw (0,0) rectangle (1200, 800);

	% pointer size 55 x 75
	\node[pointer] at (1135, 80) {};
	\node[tooltip, anchor=south east] at (1135, 80) {Algorithm I};

	\node[shadow] at (1135, 795) {};
	\node[pointer] at (1135, 725) {};
	\node[shadow] at (1135, 615) {};
	\node[tooltip, anchor=north east] at (1135, 795) {Algorithm II};


	\node[shadow] at (5, 80) {};
	\node[pointer] at (155, 80) {};
	\node[shadow] at (370, 80) {};
	\node[tooltip, anchor=south west] at (5, 80) {Algorithm III};

	\node[shadow] at (5, 795) {};
	\node[pointer] at (5, 725) {};
	\node[shadow] at (5, 615) {};
	\node[tooltip, anchor=north west] at (70, 795) {Algorithm III};


\end{tikzpicture}
\caption{Overview of the tooltip placement.}
\label{fig:tooltip_placement}
\end{figure}


\subsection{Large}
\label{sec:tooltip_placement:large}

This tooltip needs so much space that its height no longer fits on the
screen. In this case the width will be increased. There is no guarantee the
tooltip will fit properly; the creator of the tooltip should consider
whether or not it contains too much information. Once the width is adjusted the
placement algorithm used it the same as for the normal tooltip, see
\cref{sec:tooltip_placement:normal}.


\subsection{Huge}
\label{sec:tooltip_placement:huge}

The tooltip will certainly not fit nicely on the screen and no attempt is
made to even try to do something sane. Instead it is placed in the top left
corner. Unless a very small screen resolution is used the creator of the
tooltip really needs to consider reducing the amount of text on the tooltip.
There is no guarantee the entire tooltip is visible.
